\documentclass[10pt,a4paper]{article}
%\documentclass[10pt,twocolumn,a4paper]{article}

\usepackage{amsmath, amssymb, braket}
\usepackage{multicol}
\usepackage{graphicx}
\usepackage{subfig}
\usepackage{a4wide}
\usepackage[lined]{algorithm2e}
\usepackage{parcolumns}
\usepackage{mathpazo}
\usepackage{acronym}

\usepackage[colorlinks,
            pdftex,
            bookmarks=true]{hyperref}
\hypersetup{
    pdfauthor = {Giovanni Simoni}
    pdftitle = {Advanced System and Architectures --- Project report}
    pdfsubject = {IEEE 1588 PTP synchronization on an embedded platform},
    pdfkeywords = {CCS, Concurrency Workbench}
    pdfcreator = {LaTeX with hyperref package},
    pdfproducer = {pdflatex}
}

\newcommand{\Note}[1]{\paragraph{Note:}{#1}}
\newcommand{\FileName}[1]{{\sf {#1}}}
\newcommand{\Const}[1]{{\tt {#1}}}
\newcommand{\TechName}[1]{\emph{{#1}}}
\newcommand{\StdName}[1]{\emph{{#1}}}

\newcommand{\MyBoard}{\TechName{Freescale MPC8313E-RDB}}
\newcommand{\PPC}{\TechName{Power-PC}}
\newcommand{\UBoot}{\TechName{U-Boot}}
\newcommand{\Linux}{\TechName{Linux}}
\newcommand{\GNULinux}{\TechName{GNU}/\Linux}
\newcommand{\BusyBox}{\TechName{Linux}}
\newcommand{\Minicom}{\TechName{Minicom}}
\newcommand{\LTIB}{\TechName{LTIB}}

\setlength{\parindent}{0pt}
\setlength{\parskip}{2ex}

\title {
    Project report:\\
    Advanced System and Architectures\\
    IEEE 1588 PTP synchronization on an embedded platform
}
\author{
    Giovanni Simoni\\
    Register 142955\\
    \href{mailto:giovanni.simoni@roundhousecode.com}
         {giovanni.simoni@roundhousecode.com}
}

\begin{document}

\maketitle

    \section*{ \center Abstract }

        The \MyBoard\ is a \PPC-based board equiped with a
        \TechName{Vitesse VSC7385 L2 switch}, which integrates the
        \StdName{IEEE 1588} time synchronization protocol.

        TODO: check info and write stuff.

    \newpage
    \tableofcontents

    \section{ Overview }

        The \MyBoard, henceforth simply referred as \emph{board}, is
        provided with a \UBoot\ software, which by default loads a \Linux\
        kernel image with a \BusyBox\ environment. The command lines of
        both systems can be accessed through the serial port, by using a
        command like \Minicom\footnote{
            \url{http://alioth.debian.org/projects/minicom/}. Packaged for
            both \emph{Debian}-like and \emph{Red-Hat}-like distributions.
        }. In order to communicate through the serial port, an \emph{USB
        to serial} adapter can be used: on a recent enough \GNULinux\
        operating system the procedure should be totally seamless.

        The board comes along with two CD-ROMs providing the
        \TechName{CodeWarrior} IDE and some kind of loading system based
        on the \LTIB\footnote{ \url{http://www.bitshrine.org/ltib/} }
        software, however I'm a suspicious guy, and I tend to mistrust
        anything trying to simplify my life beyond a certain threshold,
        especially if they are requiring to run the \emph{rpm} command
        with \emph{root} permissions on my \emph{Debian-like} OS!

        That's why I opted for the \emph{bare-metal} approach I'm going to
        describe in this document: both \UBoot\ and \Linux\ can enable the
        Ethernet interfaces, and this can be exploited to conveniently
        load programs on the board.

        The CD-ROMs however are not totally useless, since one of them
        contains some precious files that usually nobody want to produce
        by themselves:
        \begin{itemize}

            \item   A fully-fledged cross-compiler for the \PPC\
                    architecture;

            \item   Pre-compiled binary images of the kernel, the
                    \BusyBox-based filesystem and the \emph{device tree};

            \item   All the configurations needed to recompile the
                    mentioned tools.

        \end{itemize}


\end{document}

