\documentclass[10pt,a4paper]{article}
%\documentclass[10pt,twocolumn,a4paper]{article}

\usepackage{amsmath, amssymb, braket}
\usepackage{listings}
\usepackage{multicol}
\usepackage{graphicx}
\usepackage{subfig}
\usepackage{a4wide}
\usepackage[lined]{algorithm2e}
\usepackage{parcolumns}
\usepackage{mathpazo}
\usepackage{acronym}
\usepackage{fancybox}

\usepackage[colorlinks,
            pdftex,
            bookmarks=true]{hyperref}
\hypersetup{
    pdfauthor = {Giovanni Simoni}
    pdftitle = {Advanced System and Architectures --- Project report}
    pdfsubject = {IEEE 1588 PTP synchronization on an embedded platform},
    pdfkeywords = {CCS, Concurrency Workbench}
    pdfcreator = {LaTeX with hyperref package},
    pdfproducer = {pdflatex}
}

\newcommand{\MsgBox}[3]{
    %\shadowbox{ %
    \fbox{ %
        \centering %
        \begin{minipage}{\textwidth}
        \vspace{0.3em}
        \underline{\textbf{{#1}}: {#2}}
        \vspace{1em}
        \\
        {#3}
        \vspace{0.3em}
        \end{minipage} %
    }
}

\newcommand{\Warn}[2]{\MsgBox{Warning}{#1}{#2}}
\newcommand{\Note}[2]{\MsgBox{Note}{#1}{#2}}
\newcommand{\EpicHack}[2]{\MsgBox{Epic Hack}{#1}{#2}}

\newcommand{\FileName}[1]{\textbf{\textsl{\textsf{{#1}}}}}
\newcommand{\PackageName}[1]{\textbf{\textsl{\textsf{{#1}}}}}
\newcommand{\Const}[1]{{{\tt {#1}}}}
\newcommand{\TechName}[1]{\textsf{{#1}}}
\newcommand{\ToolName}[1]{\emph{{#1}}}
\newcommand{\StdName}[1]{\emph{{#1}}}
\newcommand{\Command}[1]{{\tt {#1}}}

\newcommand{\MyBoard}{\TechName{Freescale MPC8313E-RDB}}
\newcommand{\PPC}{\TechName{Power-PC}}
\newcommand{\uBoot}{\TechName{U-Boot}}
\newcommand{\Linux}{\TechName{Linux}}
\newcommand{\GNULinux}{\TechName{GNU}/\Linux}
\newcommand{\BusyBox}{\TechName{Linux}}
\newcommand{\Minicom}{\TechName{Minicom}}
\newcommand{\LTIB}{\TechName{LTIB}}
\newcommand{\NetCat}{\TechName{Netcat}}
\newcommand{\InitRamFs}{\ToolName{initramfs}}
\newcommand{\TFTPd}{\ToolName{TFTPd}}
\newcommand{\TFTP}{\StdName{TFTP}}
\newcommand{\PTPd}{\ToolName{PTPd}}
\newcommand{\PTP}{\StdName{PTP}}
\newcommand{\BootP}{\StdName{BootP}}
\newcommand{\ARP}{\StdName{ARP}}

\setlength{\parindent}{0pt}
\setlength{\parskip}{2ex}

\title {
    Project report:\\
    Advanced System and Architectures\\
    IEEE 1588 PTP synchronization on an embedded platform
}
\author{
    Giovanni Simoni\\
    Register 142955\\
    \href{mailto:giovanni.simoni@roundhousecode.com}
         {giovanni.simoni@roundhousecode.com}
}

\begin{document}

\maketitle

    \section*{ \center Abstract }

        The \MyBoard\ is a \PPC-based board equiped with a
        \TechName{Vitesse VSC7385 L2 switch}, which integrates the
        \StdName{IEEE 1588} time synchronization protocol.

        TODO: check info and write stuff.

    \tableofcontents
    \newpage

    \section{ Overview }

        The \MyBoard, henceforth simply referred as \emph{board}, is
        provided with a \uBoot\ software, which by default loads a \Linux\
        kernel image with a \BusyBox\ environment. The command lines of
        both systems can be accessed through the serial port, by using a
        command like \Minicom\footnote{
            \url{http://alioth.debian.org/projects/minicom/}. Packaged for
        both \emph{Debian}-like and \emph{Red-Hat}-like distributions.
        }. In order to communicate through the serial port, an \emph{USB
        to serial} adapter can be used: on a recent enough \GNULinux\
        operating system the procedure should be totally seamless.

        The board comes along with two CD-ROMs providing the
        \TechName{CodeWarrior} IDE and some kind of loading system based
        on the \LTIB\footnote{ \url{http://www.bitshrine.org/ltib/} }
        software, however I'm a suspicious guy, I tend to mistrust
        anything trying to simplify my life beyond a certain threshold.
        This is true especially when the software requires to run the
        \emph{rpm} command with \emph{root} privileges on my
        \emph{Debian-like} OS!

        I ran the program into a \TechName{CentOS}-based \emph{chroot
        jail}, but I still got confused on what the software is supposed
        to do. Instead of trying understanding it I preferred to do
        some retro-engineering of the system and go for the
        \emph{bare-metal} approach I'm going to describe in this
        document.

        The idea is simple: with both \uBoot\ and \Linux\ the Ethernet
        interfaces can be enabled, and this can be exploited to
        conveniently load programs on the board. The development process
        can be achieved with the classical \emph{editor + compiler}
        combo, which is definitely the simplest way of getting stuff done.

        The CD-ROMs however are not totally useless, since one of them
        contains some precious files that usually nobody want to produce
        by themselves.

    \section{ Retrieving the tools }
    \label{sec:GetTools}

        Some useful tools and configurations are provided in the CD-ROM:
        \begin{itemize}
        \item   A fully-fledged cross-compiler for the \PPC\ architecture
                can be found in the \FileName{pkgs} directory:
                \FileName{mtwk-lnx-powerpc-gcc-3.4.3-glibc-2.3.3-0.28-1.i686.rpm};

        \item   The \FileName{images} directory contains all the binary
                objects to be loaded on the \emph{flash memory} of the board:
                \begin{itemize}
                \item   A pre-compiled binary image of the \emph{kernel}:
                        \FileName{uimage};
                \item   An image of the \emph{\BusyBox-based filesystem}:
                        \FileName{rootfs.ext2.gz.uboot};
                \item   The \emph{device tree blob}:
                        \FileName{mpc8313erdb.dtb}
                \end{itemize}
        \end{itemize}

        The configurations files they can be found inside the
        \FileName{ltib.tar.gz} tarball, in a directory named
        \FileName{config}. They can be used in case of partial or total
        rebuilding of the system.

        \Warn{}{
            If an upgrade to newer versions of the software is required,
            please mind the fact that old configuration files are usually
            not suitable for new versions of the software
        }

    \section{ Re-flashing binary images }

        The \uBoot\ software supports the network booting through \BootP\
        + \TFTP. This simplifies a lot the transfer of binary images: just
        setup servers and ask for data from the board!

        \subsection{ Network setup with \uBoot }

            The board is provided with two network interfaces: the
            \TechName{Vitesse VSC7385} and the \TechName{Marvell}. In
            \uBoot\ those are respectively enumerated as \Const{TSEC0} and
            \Const{TSEC1}, and only one of them can be activated per time:
            this setting is bound by the environment variable named
            \Const{ethact}\footnote{
                Thanks to Lorenzo and Mirko for this information!
            }.

            In order to select the \TechName{Marvell} card the following
            command is needed:
\begin{lstlisting}
    setenv ethact TSEC1
\end{lstlisting}

            \Note{Variable reset problem}{
                As specified by the \uBoot\ manual, the internal
                \uBoot\ environment can be saved by using the
                \Command{saveenv} command. This works correctly for any
                variable including \Const{ethact} but, for some reason, the
                active NIC gets re-assigned when \uBoot\ starts or gets
                reset.
            }

            During my experiments I noticed that \uBoot\ allows to set an
            IP address for the board, thus I skipped the configuration of
            the \BootP\ daemon and I relied on the default address of the
            NIC.

            The status of the network connection can be verified by using
            the simple version of the \emph{ping} program embedded by
            \uBoot: this works from the board to a host on the network,
            while the vice-versa won't work. By running a \emph{sniffer}
            I discovered that \uBoot\ don't answer to \emph{arp
            requests}, and this must be taken in account in order to
            achieve a network communication: the MAC address of the board
            must be specified on the host running the \TFTP\ server. This
            can be done by using the \Command{arp} program in this way:
\begin{lstlisting}[language=bash]
    arp -s 192.168.1.100 00:12:34:56:78
\end{lstlisting}

            \Note{Deal with \TechName{network-manager}}{
                The \TechName{network-manager} program, very common on
                recent \GNULinux\ distributions, deactivates Ethernet
                interfaces on which there's no carrier. This can be
                obnoxious, since the \emph{ARP cache} will be cleaned each
                time! You may want to deactivate
                \TechName{network-manager} or manually set the \emph{ARP}
                address each time the board is reset.
            }

        \subsection{ \TFTP\ repository }

            The \TFTPd\ daemon is packaged for \emph{Debian}-like
            distributions as \PackageName{tftpd} and \emph{Red-Hat}-like
            distributions as \PackageName{tftp-server}. In both cases it
            runs against the \TechName{xinetd} server\footnote{
                \url{http://xinetd.org}
            }, thus the configuration must be entered into the
            \FileName{/etc/xinetd.d} directory. This is the content of my
            \FileName{/etc/xinetd.d/tftp} file:
\begin{lstlisting}
    service tftp
    {
        id = tftp-dgram
        type = UNLISTED
        disable = no
        socket_type = dgram
        protocol = udp
        wait = yes
        server = /usr/sbin/in.tftpd
        server_args = -s /opt/tftpd
        per_source = 1
        user = root
        port = 69
    }
\end{lstlisting}
            where \FileName{/opt/tftpd} is the directory where I stored
            all the binary objects mentioned in
            Section~\ref{sec:GetTools}.

            Once the server has been configured, the \TechName{xinetd}
            daemon must be restarted in order for the modifications to
            take place.

        \subsection{ Installing binary objects }

    \section{ Program upload techniques }

        \subsection{ Quick \& Ephemeral }

        \subsection{ Slow \& Permanent }

            \subsubsection{ Hacking the filesystem image }

            \subsubsection{ Rebuilding the \InitRamFs }

    \section{ The \PTP\ daemon }

    \section{ Some useful tricks }

        \subsection{ Images checking }

            When I first analyzed the original system on the board, I
            wasn't sure about which of the binary images were stored
            inside the flash memory.

            Of course the file names are self-explanatory enough, but the
            CD-ROM provides many different files of the same type. There
            are, for instance, many \emph{device tree blobs}:
            \begin{itemize}
                \item \FileName{mpc8313erdb.dtb}
                \item \FileName{mpc8313erdb\_usbgadget\_external\_phy.dtb}
                \item \FileName{mpc8313erdb\_usbgadget\_internal\_phy.dtb}
                \item \FileName{mpc8313erdb\_usbhost\_external\_phy.dtb}
                \item \FileName{mpc8313erdb\_usbotg\_external\_phy.dtb}
            \end{itemize}
            How to choose the right one?

            The \uBoot\ system provides a command named
            \Command{crc}\footnote{
                \url{http://www.denx.de/wiki/view/DULG/UBootCmdGroupMemory\#Section\_5.9.2.2.}
            }
            which as computes the \StdName{CRC-32 checksum} of a certain
            memory area.

            The size of all \emph{.dtb} files is 12 Kilobytes
            (\Const{0x3000} in hexadecimal) thus, once
            the address of the installed \emph{device tree} is known, its
            \StdName{CRC} can be obtained as follows:
\begin{lstlisting}
    => crc fe300000 0x3000
    CRC32 for fe300000 ... fe302fff ==> cb588b5e
\end{lstlisting}

            Now we can use a checksum program (like \Command{jacksum}) to
            obtain the \StdName{CRC-32} of all images:
\begin{lstlisting}
    $ for i in *dtb; do jacksum -a crc32 $i; done
    2678061774	12288	mpc8313erdb.dtb
    1614041190	12288	mpc8313erdb_usbgadget_external_phy.dtb
    161761636	12288	mpc8313erdb_usbgadget_internal_phy.dtb
    2371930621	12288	mpc8313erdb_usbhost_external_phy.dtb
    3656584488	12288	mpc8313erdb_usbotg_external_phy.dtb
\end{lstlisting}
            If we convert in decimal the value extracted from \uBoot\ we
            get:
\begin{lstlisting}
    $ echo 'print 0x9f9fface, "\n"' | perl
    2678061774
\end{lstlisting}
            which corresponds to the \FileName{mpc8313erdb.dtb} file.

            This trick can be used to check consistency of uploaded
            images.

\end{document}

