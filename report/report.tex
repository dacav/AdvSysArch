\documentclass[10pt,a4paper]{article}
%\documentclass[10pt,twocolumn,a4paper]{article}

\usepackage{amsmath, amssymb, braket}
\usepackage{listings}
\usepackage{multicol}
\usepackage{graphicx}
\usepackage{subfig}
\usepackage{a4wide}
\usepackage[lined]{algorithm2e}
\usepackage{parcolumns}
\usepackage{mathpazo}
\usepackage{acronym}
\usepackage{fancybox}

\usepackage[colorlinks,
            pdftex,
            bookmarks=true]{hyperref}
\hypersetup{
    pdfauthor = {Giovanni Simoni}
    pdftitle = {Advanced System and Architectures --- Project report}
    pdfsubject = {IEEE 1588 PTP synchronization on an embedded platform},
    pdfkeywords = {CCS, Concurrency Workbench}
    pdfcreator = {LaTeX with hyperref package},
    pdfproducer = {pdflatex}
}

\setlength{\parindent}{0pt}
\setlength{\parskip}{2ex}



\newcommand{\MsgBox}[3]{
    %\shadowbox{ %
    \fbox{ %
        \centering %
        \begin{minipage}{\textwidth}
        \vspace{0.3em}
        \underline{\textbf{{#1}}: {#2}}
        \vspace{1em}
        \\
        {#3}
        \vspace{0.3em}
        \end{minipage} %
    }
}

\newcommand{\Warn}[2]{\MsgBox{Warning}{#1}{#2}}
\newcommand{\Note}[2]{\MsgBox{Note}{#1}{#2}}
\newcommand{\EpicHack}[2]{\MsgBox{Epic Hack}{#1}{#2}}

\newcommand{\FileName}[1]{\textbf{\textsl{\textsf{{#1}}}}}
\newcommand{\PackageName}[1]{\textbf{\textsl{\textsf{{#1}}}}}
\newcommand{\Const}[1]{{{\tt {#1}}}}
\newcommand{\TechName}[1]{\textsf{{#1}}}
\newcommand{\ToolName}[1]{\emph{{#1}}}
\newcommand{\StdName}[1]{\emph{{#1}}}
\newcommand{\Command}[1]{{\tt {#1}}}

\newcommand{\MyBoard}{\TechName{Freescale MPC8313E-RDB}}
\newcommand{\PPC}{\TechName{Power-PC}}
\newcommand{\uBoot}{\TechName{U-Boot}}
\newcommand{\Linux}{\TechName{Linux}}
\newcommand{\GNULinux}{\TechName{GNU}/\Linux}
\newcommand{\BusyBox}{\TechName{Busybox}}
\newcommand{\mkimage}{\TechName{mkimage}}
\newcommand{\Minicom}{\TechName{Minicom}}
\newcommand{\LTIB}{\TechName{LTIB}}
\newcommand{\NetCat}{\TechName{Netcat}}
\newcommand{\InitRamFs}{\ToolName{initramfs}}
\newcommand{\TFTPd}{\ToolName{TFTPd}}
\newcommand{\TFTP}{\StdName{TFTP}}
\newcommand{\PTPd}{\ToolName{PTPd}}
\newcommand{\BootPd}{\ToolName{BootP}}
\newcommand{\PTP}{\StdName{PTP}}
\newcommand{\BootP}{\StdName{BootP}}
\newcommand{\ARP}{\StdName{ARP}}

\newcommand{\Debian}{\TechName{Debian}}
\newcommand{\RedHat}{\TechName{Red-hat}}
\newcommand{\SourceForge}{\TechName{Sourceforge.net}}


\title {
    Project report:\\
    Advanced System and Architectures\\
    IEEE 1588 PTP synchronization on an embedded platform
}
\author{
    Giovanni Simoni\\
    Register 142955\\
    \href{mailto:giovanni.simoni@roundhousecode.com}
         {giovanni.simoni@roundhousecode.com}
}

\begin{document}

\maketitle

    \section*{ \center Abstract }

        \StdName{IEEE 1588} is a standard defining the \emph{Precision
        Time Protocol}, a protocol which allows to synchronize the
        computers on a LAN.

        The \MyBoard\ is a \PPC-based board equipped with a
        \TechName{Vitesse VSC7385 L2 switch}. This device integrates in
        the logic required to propagate the \PTP\ datagrams in hardware,
        reducing so far the latency to a couple of \emph{nanoseconds}.

        In this report I'll illustrate how to run \PTPd, a
        \emph{BSD-licensed} software which understands the protocol and
        controls the hardware.

    \tableofcontents
    \newpage

    \section{ Overview } \label{sec:Overview}
    The \MyBoard\ is provided with a \uBoot\ software, which by default loads
a \Linux\ kernel image with a \BusyBox\ environment. The command lines of
both systems can be accessed through the serial port. In order to
communicate through the serial port, an \emph{USB to serial} adapter can
be used: on a recent enough \GNULinux\ operating system the procedure
should be totally seamless.

The board comes along with two CD-ROMs providing the
\TechName{CodeWarrior} IDE and some kind of loading system based on the
\LTIB\footnote{ \url{http://www.bitshrine.org/ltib/} } software, however
I'm a suspicious guy, I tend to mistrust anything trying to simplify my
life beyond a certain threshold.  This is true especially when the
software requires to run the \emph{rpm} command with \emph{root}
privileges on my \Debian-like OS!

I ran the program into a \TechName{CentOS} \emph{chroot jail}, but I
still got confused on what the software is supposed to do. Instead of
trying understanding it I preferred to do some retro-engineering of the
system and go for the \emph{bare-metal} approach I'm going to describe in
this document.

The idea is simple: with both \uBoot\ and \Linux\ the Ethernet interfaces
can be enabled, and this can be exploited to conveniently load programs on
the board. The development process can be achieved with the classic
\emph{editor + compiler} combo, which in my opinion is definitely the
simplest way of getting stuff done.

The CD-ROMs however are not totally useless, since one of them contains
some precious files that usually nobody want to produce by themselves.
More on that in the next section.

\subsection{ Naming conventions of this document }

    \begin{itemize}

    \item   The \MyBoard\ will be simply referred as
            ``\emph{board}'';

    \item   The personal computer or laptop on which compilations and
            configurations are achieved is referred as ``\emph{host}'';

    \item   Listings to be executed on the host will have the
            ``\lstinline{host$}'' prompt:
\begin{lstlisting}
    host$ echo Hello
\end{lstlisting}

    \item   Listings to be executed on the board with the
            \BusyBox\ + \Linux\ environment will have the
            ``\lstinline{board$}'' prompt:
\begin{lstlisting}
    board$ echo Hello
\end{lstlisting}

    \item   Listings to be executed on the board with the \uBoot
            system will have the ``\lstinline{=>}'' prompt:
\begin{lstlisting}
    => echo Hello
\end{lstlisting}

    \end{itemize}

\subsection{ Resources } \label{sub:Resources}

    During my laboratory experience I produced a bunch of files
    (configurations, patches, this document \dots). Everything has been
    stored on my \TechName{Github} repository at
    \url{https://github.com/dacav/AdvSysArch}.

\subsection{ What are we running? }

    During the process realization I aimed at time efficiency:
    instead of producing from scratch all the tools I preferred to
    hack and modify the existent environment.

    The approach is reasonable, since the system was not that
    outdated as I initially thought, anyways one might want to
    update it. The information contained in this document can be
    very useful also for this kind of operation.



    \section{ Retrieving the tools } \label{sec:GetTools}
    As mentioned in Section~\ref{sec:Overview} useful tools and configurations
are provided in the CD-ROM:
\begin{itemize}
\item   A fully-fledged cross-compiler for the \PPC\ architecture
        can be found in the \FileName{pkgs} directory:
        \FileName{mtwk-lnx-powerpc-gcc-3.4.3-glibc-2.3.3-0.28-1.i686.rpm};

\item   The \FileName{images} directory contains all the binary
        objects to be loaded on the \emph{flash memory} of the
        board:
        \begin{itemize}
        \item   A pre-compiled binary image of the \emph{kernel}:
                \FileName{uimage};
        \item   An image of the \emph{\BusyBox-based filesystem}:
                \FileName{rootfs.ext2.gz.uboot};
        \item   The \emph{device tree blob}:
                \FileName{mpc8313erdb.dtb}
        \end{itemize}
\end{itemize}

The configurations files they can be found inside the
\FileName{ltib.tar.gz} tarball, in a directory named \FileName{config}.
They can be used in case of partial or total rebuilding of the system.

\Warn{}{
    If an upgrade to newer versions of the software is required, please
    mind the fact that old configuration files are usually not suitable
    for new versions of the software
}




    \section{ A simple program uploading technique } \label{sec:Upload}
    When I booted the board with the untampered factory settings I noticed
that the installed \BusyBox\ + \Linux\ system is provided with a
\NetCat-like tool\footnote{\url{http://nc110.sourceforge.net/}}.

The very first test I ran was trying to execute a simple
\emph{Hello World} C program on the board as follows:
\begin{enumerate}

    \item   Cross-compile the program:

\begin{lstlisting}
    host$ cat > hello.c << EOF
    > #include <stdio.h>
    > int main ()
    > {
    >     printf("Hello world!\n");
    >     return 0;
    > }
    > EOF
    host$ $ CC=powerpc-linux-gcc make hello
    powerpc-linux-gcc   hello.o   -o hello
\end{lstlisting}

    \item   Feed a \NetCat\ in server mode with the executable:

\begin{lstlisting}
    host$ nc -l 9000 < hello
\end{lstlisting}

    \item   Retrieve the program from the board and execute it:

\begin{lstlisting}
    board$ nc 192.168.1.1 > hello
    board$ chmod +x ./hello
    board$ ./hello
    Hello world
\end{lstlisting}

\end{enumerate}

\Warn{Interrupt from keyboard}{
    The installed \BusyBox\ + \Linux\ system may not support the
    \emph{termination from keyboard} sequence (\Command{CTRL+C}); this is
    probably due to the serial interface. It's a good practice to launch
    programs like \NetCat, which waits for some event, as \emph{background
    process}.

    Also if you plan to test the network with \Command{ping}, for the same
    reason I suggest to use the \Command{-c} option in order to get a
    bounded execution.
}

This technique is a very simple way of running a program on the board, but
since the filesystem works in RAM, the program will be lost at the first
board reboot. On the other hand this doesn't require to manipulate the
system image stored in the flash memory, so it's very quick.

By using this trick you should be able to achieve software testing in a
convenient way, while the flashing operation can be delayed to the
deployment phase.




    \section{ Hacking binary images } \label{sec:HackImages}
    After a proper testing phase one may want to deploy the application in a
stable environment, possibly with some automatisms.

An Embedded \Linux\ environment needs three main items: \emph{kernel}, a
\emph{filesystem image} and a \emph{device tree}. Those three guys, as
explained in Section~\ref{sec:GetTools}, are provided by
\TechName{Freescale}.

If we want to permanently install programs on the board we basically need
to modify the filesystem, while the other two elements can be used as they
are (as long as the functionality we need aren't placed in kernel space).

\subsection{The filesystem image}

    The filesystem is named \FileName{rootfs.ext2.gz.uboot}, thus what I
    expected is just a \emph{gzipped} \StdName{ext2} filesystem enveloped
    into some \uBoot\ wrapper:

\begin{lstlisting}
    host$ file rootfs.ext2.gz.uboot
    rootfs.ext2.gz.uboot: u-boot legacy uImage, uboot ext2 ramdisk rootfs,
    Linux/PowerPC, RAMDisk Image (gzip), 2831355 bytes, Fri Aug 24
    17:01:41 2007, Load Address: 0x00000000, Entry Point: 0x00000000,
    Header CRC: 0x9B7D6AEB, Data CRC: 0x14B719EB
\end{lstlisting}

    Searching on the web I've found out that this kind of image is
    produced by the \mkimage\ tool, which envelops stuff inside a 64 bytes
    header. Not a big deal for our friend \Command{dd}, right?

    So this is an example shell session you can use to mount the
    filesystem on some directory:
\begin{lstlisting}
    host$ dd if=rootfs.ext2.gz.uboot bs=64 skip=1 of=rootfs.ext2.gz
    44239+1 records in
    44239+1 records out
    2831355 bytes (2.8 MB) copied, 0.0881506 s, 32.1 MB/s
    host$ gunzip rootfs.ext2.gz
    host$ mkdir userland
    host$ su -c 'mount -t ext2 -o rw,loop rootfs.ext2 userland'
    Password:
    host$ ls userland/
    bin/  etc/   lib/      lost+found/  opt/   root/  sys/  usr/
    dev/  home/  linuxrc@  mnt/         proc/  sbin/  tmp/  var/
\end{lstlisting}
    Bingo!

    \Warn{Using \Command{su} or \Command{sudo}}{
        Depending on your distribution's policy, the execution of a
        command with \emph{root} privileges could require the \Command{su}
        or the \Command{sudo} command. For instance, if you are on a
        \TechName{Ubuntu} system you are likely to be using \Command{sudo}
        for the \Command{mount} command:
        \begin{quote}\texttt{
            host\$ sudo mount -t ext2 -o rw,loop rootfs.ext2 userland
        }
        \end{quote}
    }

    Now we need to install some program, say the \emph{hello world}
    program we compiled in Section~\ref{sec:Upload}, inside the image. We
    just need to copy it:

\begin{lstlisting}
    host$ cp /tmp/hello userland/bin
\end{lstlisting}

\subsection{The startup script}

    At this point we need a little magic: during bootstrap the \Linux\
    kernel relies on a temporary root filesystem, which usually contains
    the modules to be loaded before starting the actual operating system.
    When the modules are loaded the \Command{pivot\_root} command is used
    to swap the temporary root with the actual root.

    Many different solutions are supported here, but the
    \StdName{initramfs} is definitely the most easy and immediate: it's
    shaped as a \StdName{gzipped cpio} archive. More information on this
    topic can be easily found on the web.

    After the bootstrap, the \Linux\ kernel will try to execute the file
    named \FileName{/init}, so what we want to do is skip the root
    pivoting and remain in the \StdName{initramfs} environment. If we want
    to automatically execute the \emph{hello world} program at bootstrap
    we just need something like that:

\begin{lstlisting}
    host$ cat > userland/init << EOF
    > #!/bin/ash
    >
    > mount -t proc /proc proc
    > mount -t sysfs none /sys
    >
    > hello
    >
    > echo 'Ready to go! Welcome!'
    > /bin/ash --login
    > EOF
\end{lstlisting}

    Now we can build an \StdName{initramfs} image starting from the
    modified filesystem by using the \Command{cpio} command, and
    subsequently envelop it into the wrapper for \uBoot\ by using mkimage

\begin{lstlisting}
    host$ cd userland
    host$ find . | cpio -o -H newc | gzip --best > ../initramfs.img
    host$ cd ..
    host$ mkimage -A powerpc -O Linux -T ramdisk -C gzip -n Sylvie -a 0 -e
          0 -d initramfs.img rootfs.hello.uboot

    Image Name:   Sylvie
    Created:      Wed Jul 27 17:31:47 2011
    Image Type:   PowerPC Linux RAMDisk Image (gzip compressed)
    Data Size:    2800916 Bytes = 2735.27 kB = 2.67 MB
    Load Address: 00000000
    Entry Point:  00000000
\end{lstlisting}

    Our image, the file \FileName{rootfs.hello.uboot} is now ready to be
    loaded on the board. This aspects are covered in
    Section~\ref{sec:UploadImages}.



    \section{ Uploading binary images } \label{sec:UploadImages}
    The \uBoot\ software supports the network booting through \BootP\ + \TFTP.
This functionality is extremely useful, since it allows to transfer a
binary object, like a \emph{filesystem image} or a \emph{kernel}, from a
server to the board.

The default system, installed by \TechName{Freescale}, defines a boot
procedure which loads an embedded operating system from the flash memory.
I modified such setting, and now it behaves as follows:

\begin{enumerate}

\item   After a basic bootstrap the board queries the network with the
        \StdName{dhcp};

\item   The host answers providing:
    \begin{itemize}
    \item   An IP address;
    \item   The name of a \uBoot\ script to be loaded and executed.
    \end{itemize}

\item   Basing on the script two different things can happen:
    \begin{itemize}
    \item   The default system booting procedure is activated
            (\emph{normal booting});
    \item   A new \emph{initramfs} is loaded into memory and copied inside
            the flash memory, replacing the previous image of the
            filesystem (\emph{update booting}).
    \end{itemize}
    In both cases the assigned IP number is propagated also to the \Linux\
    system trought the kernel's command line options.

\end{enumerate}

This procedure particularly well suited for a network environment: by
modifying the settings of the \BootP\ + \TFTP\ server it's possible to
redefine the behavior of the deployed boards in a centralized way.

Many tutorials on the Internet cover in detail how to properly configure
the daemons. This section contains the dump of my configuration files and
an explanation on how I implemented my idea.

\subsection{ Network setup }

    The board is provided with two network interfaces: the
    \TechName{Vitesse VSC7385} and the \TechName{Marvell}. In
    \uBoot\ those are respectively enumerated as \Const{TSEC0} and
    \Const{TSEC1}, and only one of them at a time can be
    activated. This setting is bound by the environment variable
    named \Const{ethact}.

    In order to select the \TechName{Marvell} card the following
    command is needed:
\begin{lstlisting}
    => setenv ethact TSEC1
\end{lstlisting}

    \Warn{Variable reset problem}{
        As specified by the \uBoot\ manual, the internal
        \uBoot\ environment can be saved by using the
        \Command{saveenv} command. This works correctly for any
        variable including \Const{ethact}, but for some reason
        the active NIC gets re-assigned when \uBoot\ starts or
        gets reset.
    }

    During my experiments I noticed that \uBoot\ allows to both set
    manually an IP address for the board or use a \BootP\ server in order
    to make the procedure automated. The status of the network connection
    can be verified by using a simple version of the \emph{ping} program
    embedded into \uBoot.

    By running a \emph{sniffer} I discovered that \uBoot\ doesn't answer
    to \emph{arp requests}: the MAC address must be setted manually on the
    host. This problem doesn't show up if a \BootP\ server is used.

    \Note{How to deal with \TechName{network-manager}}{
        The \TechName{network-manager} program, very common on
        recent \GNULinux\ distributions, deactivates Ethernet
        interfaces on which there's no carrier. This can be
        obnoxious, since the \emph{ARP cache} will be cleaned each
        time you reset the board. You may want to temporarily
        deactivate \TechName{network-manager} or manually set the
        \emph{ARP} address when needed.
    }

\subsection{ \BootPd\ and \TFTPd } \label{sub:Xinetd}

    Both the choosen versions of the packages (see
    Subsection~\ref{sub:OtherTools}) rely on \TechName{Xinetd}\footnote{
        \url{http://xinetd.org}
    }, a secure replacement for the \TechName{Inet} daemon.

    \subsubsection{\TFTPd}

        The \TFTPd\ configuration is trivial as it should be. We just need
        to tell \TechName{Xinetd} how to start the daemon and we are
        done:

\begin{lstlisting}
    host$ more /etc/xinetd.d/tftp
    service tftp
    {
        id = tftp-dgram
        type = UNLISTED
        disable = no
        socket_type = dgram
        protocol = udp
        wait = yes
        server = /usr/sbin/in.tftpd
        server_args = -s /opt/tftpd
        per_source = 1
        user = root
        port = 69
    }
\end{lstlisting}

        Some comments:
        \begin{itemize}
        \item   My own setting uses the directory \FileName{/opt/tftpd}
                has been chosen as root for the server. All the binaries
                mentioned in Section~\ref{sec:GetTools} are stored there;
        \item   The name of the program to be executed is
                \Command{in.tftpd};
        \item   The protocol works on UDP port 69 (according to the
                well-known ports, see \FileName{/etc/services}).
        \end{itemize}

    \subsubsection{\BootPd}

        The configuration of the \BootPd\ daemon is mainly splitted among
        \TechName{Xinetd} settings and \BootPd's own configuration file:
        \FileName{/etc/bootptab}.

        The former is pretty general, as for \TFTPd:
\begin{lstlisting}
    host$ more /etc/xinetd.d/bootp
    service bootps
    {
        id = bootp-dgram
        type = UNLISTED
        disable = no
        socket_type = dgram
        protocol = udp
        wait = yes
        user = root
        server = /usr/sbin/bootpd
        per_source = 1
        port = 67
    }
\end{lstlisting}

        My \FileName{/etc/bootptab} is the following:
\begin{lstlisting}
    host$ more /etc/boottab
    .default:\
        :sm=255.255.255.0:\
        :sa=192.168.1.1:\
        :gw=192.168.1.1

    .noupdate:\
        :tc=.default:\
        :bf=bootseq_run.img

    .update:\
        :tc=.default:\
        :bf=bootseq_update.img

    board:\
        :ht=ether:\
        :ha=00fa10fafa10:\
        :tc=.update:\
        :ip=192.168.1.200
\end{lstlisting}

        Some comments:
        \begin{itemize}

        \item   The \Const{.default} label identifies general settings of
                the network;

        \item   The \Const{tc} keyword indicates inheritance of the
                settings from another label;

        \item   The two behaviors described at the very beginning of this
                section are managed trough the \Const{.noupdate} and the
                \Const{.update} labels:

            \begin{enumerate}
            \item   The \Const{.noupdate} label defines as boot file the
                    script which simply runs the system as it is;
            \item   The \Const{.update} label defines as boot file the
                    script which first updates the system and then runs
                    it.
            \end{enumerate}

        \item   Finally, for each board (in my case just one of
                them, I named it just ``\emph{board}'') we can define the
                IP address and the kind of operation the board will do on
                bootstrap (in this case the system will be updated).

        \end{itemize}



%    \section{ The \PTP\ daemon } \label{sec:PTPd}

    \section{ Some useful tricks } \label{sec:Tricks}
    \subsection{ Images checking } \label{sub:ImagesChecking}

    When I first analyzed the original system on the board, I
    wasn't sure about which of the binary images were stored
    inside the flash memory.

    Of course the file names are self-explanatory enough, but the
    CD-ROM provides many different files of the same type. There
    are, for instance, many \emph{device tree blobs}:
    \begin{itemize}
        \item \FileName{mpc8313erdb.dtb}
        \item \FileName{mpc8313erdb\_usbgadget\_external\_phy.dtb}
        \item \FileName{mpc8313erdb\_usbgadget\_internal\_phy.dtb}
        \item \FileName{mpc8313erdb\_usbhost\_external\_phy.dtb}
        \item \FileName{mpc8313erdb\_usbotg\_external\_phy.dtb}
    \end{itemize}
    How to choose the right one?

    The \uBoot\ system provides a command named
    \Command{crc}\footnote{
        \url{http://www.denx.de/wiki/view/DULG/UBootCmdGroupMemory\#Section\_5.9.2.2.}
    }
    which as computes the \StdName{CRC-32 checksum} of a certain
    memory area.

    The size of all \emph{.dtb} files is 12 Kilobytes
    (\Const{0x3000} in hexadecimal) thus, once
    the address of the installed \emph{device tree} is known, its
    \StdName{CRC} can be obtained as follows:
\begin{lstlisting}
    => crc fe300000 0x3000
    CRC32 for fe300000 ... fe302fff ==> cb588b5e
\end{lstlisting}

    Now we can use a checksum program (like \Command{jacksum}) to
    obtain the \StdName{CRC-32} of all images:
\begin{lstlisting}
    host$ for i in *dtb; do jacksum -a crc32 $i; done
    2678061774	12288	mpc8313erdb.dtb
    1614041190	12288	mpc8313erdb_usbgadget_external_phy.dtb
    161761636	12288	mpc8313erdb_usbgadget_internal_phy.dtb
    2371930621	12288	mpc8313erdb_usbhost_external_phy.dtb
    3656584488	12288	mpc8313erdb_usbotg_external_phy.dtb
\end{lstlisting}
    If we convert in decimal the value extracted from \uBoot\ we
    get:
\begin{lstlisting}
    host$ echo 'print 0x9f9fface, "\n"' | perl
    2678061774
\end{lstlisting}
    which corresponds to the \FileName{mpc8313erdb.dtb} file.

    This trick can be used to check consistency of uploaded
    images.



%    \section{ Updating the system } \label{sec:Update}

\end{document}

