The \uBoot\ software supports the network booting through \BootP\ + \TFTP.
This allows to transfer a binary object, like a \emph{filesystem image} or
a \emph{kernel}, from a server to the board.

The default system, installed by \TechName{Freescale}, defines a boot
procedure which loads an embedded operating system from the flash memory.
I modified such setting, and now it behaves as follows:

\begin{enumerate}

\item   After a basic bootstrap the board queries the network with the
        \StdName{dhcp};

\item   The host answers providing:
    \begin{itemize}
    \item   The network configuration (IP address + subnet mask);
    \item   The name of a \uBoot\ script to be loaded and executed.
    \end{itemize}

\item   Basing on the script two different things can happen:
    \begin{itemize}
    \item   The default system booting procedure is activated
            (\emph{normal booting});
    \item   A new \emph{initramfs} is loaded into memory and copied inside
            the flash memory, replacing the previous image of the
            filesystem (\emph{update booting}).
    \end{itemize}
    In both cases the assigned IP number is propagated also to the \Linux\
    system trought the kernel's command line options.

\end{enumerate}

This procedure particularly well suited for a network environment: by
modifying the settings of the \BootP\ + \TFTP\ server it's possible to
redefine the behavior of the deployed boards in a centralized way.

Many tutorials on the Internet cover in detail how to properly configure
the daemons. This section contains the dump of my configuration files and
an explanation on how I implemented my idea.

But first of all a little overview on how to mess with the network\dots

\subsection{ Network setup }

    The board is provided with two network interfaces: the
    \TechName{Vitesse VSC7385} and the \TechName{Marvell}. In
    \uBoot\ those are respectively enumerated as \Const{TSEC0} and
    \Const{TSEC1}, and only one of them at a time can be
    activated. This setting is bound by the environment variable
    named \Const{ethact}.

    In order to select the \TechName{Marvell} card the following
    command is needed:
\begin{lstlisting}
    => setenv ethact TSEC1
\end{lstlisting}

    \Warn{Variable reset problem}{
        As specified by the \uBoot\ manual, the internal
        \uBoot\ environment can be saved by using the
        \Command{saveenv} command. This works correctly for any
        variable including \Const{ethact}, but for some reason
        the active NIC gets re-assigned when \uBoot\ starts or
        gets reset.
    }

    During my experiments I noticed that \uBoot\ allows to both set
    manually an IP address for the board or use a \BootP\ server in order
    to make the procedure automated. The status of the network connection
    can be verified by using a simple version of the \emph{ping} program
    embedded into \uBoot.

    By running a \emph{sniffer} I discovered that \uBoot\ doesn't answer
    to \emph{arp requests}: the MAC address must be set manually on the
    host. This problem doesn't show up if a \BootP\ server is used.

    \Note{How to deal with \TechName{network-manager}}{
        The \TechName{network-manager} program, very common on
        recent \GNULinux\ distributions, deactivates Ethernet
        interfaces on which there's no carrier. This can be
        obnoxious, since the \emph{ARP cache} will be cleaned each
        time you reset the board. You may want to temporarily
        deactivate \TechName{network-manager} or manually set the
        \emph{ARP} address when needed.
    }

\subsection{ \BootPd\ and \TFTPd } \label{sub:Xinetd}

    Both the choosen versions of the packages (see
    Subsection~\ref{sub:OtherTools}) rely on \TechName{Xinetd}\footnote{
        a secure replacement for the \TechName{Inet} daemon.
        \url{http://xinetd.org}
    }

    \subsubsection{\TFTPd}

        The \TFTPd\ configuration is trivial as it should be. We just need
        to tell \TechName{Xinetd} how to start the daemon and we are
        done:

\begin{lstlisting}
    host$ more /etc/xinetd.d/tftp
    service tftp
    {
        id = tftp-dgram
        type = UNLISTED
        disable = no
        socket_type = dgram
        protocol = udp
        wait = yes
        server = /usr/sbin/in.tftpd
        server_args = -s /opt/tftpd
        per_source = 1
        user = root
        port = 69
    }
\end{lstlisting}

        Some comments:
        \begin{itemize}
        \item   My own setting uses the directory \FileName{/opt/tftpd}
                has been chosen as root for the server. All the binaries
                mentioned in Section~\ref{sec:GetTools} are stored there;
        \item   The name of the program to be executed is
                \Command{in.tftpd};
        \item   The protocol works on UDP port 69 (according to the
                well-known ports, see \FileName{/etc/services}).
        \end{itemize}

    \subsubsection{ \BootPd } \label{subsub:BootPd}

        The configuration of the \BootPd\ daemon is mainly splitted among
        \TechName{Xinetd} settings and \BootPd's own configuration file:
        \FileName{/etc/bootptab}.

        The former is pretty general, as for \TFTPd:
\begin{lstlisting}
    host$ more /etc/xinetd.d/bootp
    service bootps
    {
        id = bootp-dgram
        type = UNLISTED
        disable = no
        socket_type = dgram
        protocol = udp
        wait = yes
        user = root
        server = /usr/sbin/bootpd
        per_source = 1
        port = 67
    }
\end{lstlisting}
        while \FileName{/etc/bootptab} contains the actual loading system
        logic. My one is the following:
\begin{lstlisting}
    host$ more /etc/boottab
    .default:\
        :sm=255.255.255.0:\
        :sa=192.168.1.1:\
        :gw=192.168.1.1

    .noupdate:\
        :tc=.default:\
        :bf=bootseq_run.img

    .update:\
        :tc=.default:\
        :bf=bootseq_update.img

    board:\
        :ht=ether:\
        :ha=00fa10fafa10:\
        :tc=.update:\
        :ip=192.168.1.200
\end{lstlisting}

        Some comments:
        \begin{itemize}

        \item   The \Const{.default} label identifies general settings of
                the network;

        \item   The \Const{tc} keyword indicates inheritance of the
                settings from another label;

        \item   The two behaviors, described at the very beginning of this
                section, are managed trough the \Const{.noupdate} and the
                \Const{.update} labels. \BootP\ allows to define a
                \emph{boot file} which is automatically loaded by the
                \emph{client}:
            \begin{enumerate}
            \item   The \Const{.noupdate} label defines as boot file the
                    script which simply runs the system as it is;
            \item   The \Const{.update} label defines as boot file the
                    script which first updates the system and then runs
                    it.
            \end{enumerate}
                Further details on Subsection~\ref{sub:StartupModes}.

        \item   Finally, for each board (in my case just one of
                them, I named it just ``\emph{board}'') we can define the
                IP address and the kind of operation the board will do on
                bootstrap (in this case the system will be updated).

        \end{itemize}

\subsection{ Startup modes } \label{sub:StartupModes}

    \subsubsection{Bootstrap modification}

        \uBoot\ conventionally uses, as default bootstrap procedure, the
        sequence of operations listed inside the \Const{bootcmd}
        environment variable.

        The default \uBoot\ setting by \TechName{Freescale} sets
        \Const{bootcmd} as follows:
\begin{lstlisting}
    bootcmd=run run_vscld1; run ramargs addtty;bootm fe100000 fe300000
      fe700000
    run_vscld=tftp 40000 /tftpboot/vsc7385_load.bin;go 40004 
    ramargs=setenv bootargs root=/dev/ram rw
    addtty=setenv bootargs ${bootargs} console=ttyS0,${baudrate}
    baudrate=115200
\end{lstlisting}

        Basically what they are doing is just loading the operating system
        having the kernel flashed in location \Const{0xfe100000}, the
        filesystem image in location \Const{0xfe300000} and the device
        tree at \Const{0xfe700000}.

        I moved such sequence into another variable named
        \Const{vendor\_bootcmd}:
\begin{lstlisting}
    => setenv vendor_bootcmd ${bootcmd}
\end{lstlisting}
    Then I modified the \Const{bootcmd} environment variable as follows:
\begin{lstlisting}
    => setenv bootcmd "setenv loadaddr 0x100000; dhcp; source 0x100000"
\end{lstlisting}

        Do you remember the files I mentioned in
        Paragraph~\ref{subsub:BootPd}? In the \FileName{/etc/boottab}
        configuration, \FileName{bootseq\_update.img} and
        \FileName{bootseq\_run.img} are set as \emph{boot file} for the
        \Const{.update} and the \Const{.noupdate} label respectively.

        This is what happens with the new setting:
        \begin{enumerate}
        \item   The \Const{loadaddr} variable is set to a certain RAM
                address (I chose arbitrarily \Const{0x100000}). Such
                variable is conventionally used by the \uBoot\
                \Command{tftp} command as target memory area for the image
                loaded from the network;
        \item   The \Const{dhcp} command loops on the network interfaces
                and probes for a \BootP\ service. The \emph{boot file}
                pointed by \BootP\ is downloaded from the \TFTP\ server;
        \item   The \Command{source} command executes the code placed at
                location \Const{0x100000} (i.e. where the image has been
                loaded).
        \end{enumerate}

    \subsubsection{The internal scripts}

        At this point the question is: what are those two files supposed to
        do? A nice functionality of \uBoot\ is the support for a basic
        scripting system. Script files are encapsulated inside binary
        images with the \mkimage\ program.

        \FileName{bootseq\_run} just requires the execution of the default
        bootstrap procedure:
\begin{lstlisting}
    host$ more bootseq_run
    run vendor_bootcmd
\end{lstlisting}

        \FileName{bootseq\_update} loads the \FileName{rootfs.ptpd.uboot}
        image from the \TFTP\ server and writes it into the flash memory,
       exactly where the default bootstrap procedure expects to find it.
        After that the default bootstrap procedure is executed.
\begin{lstlisting}
    host$ more bootseq_update
    setenv kernel 0xfe100000
    setenv initramfs 0xfe300000
    setenv devtree 0xfe700000
    setenv filesize 0

    tftp ${loadaddr} /rootfs.ptpd.uboot
    erase ${initramfs} +0x${filesize}
    cp.b 0x${loadaddr} ${initramfs} 0x${filesize}

    run vendor_bootcmd
\end{lstlisting}

