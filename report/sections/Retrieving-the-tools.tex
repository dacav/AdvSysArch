As mentioned in Section~\ref{sec:Overview} useful tools and configurations
are provided in the CD-ROM:
\begin{itemize}

\item   A fully fledged cross-compiler for the \PPC\ architecture
        can be found in the \FileName{pkgs} directory:
        \FileName{mtwk-lnx-powerpc-gcc-3.4.3-glibc-2.3.3-0.28-1.i686.rpm};

\item   The \FileName{images} directory contains all the binary
        objects to be loaded on the \emph{flash memory} of the
        board:
        \begin{itemize}
        \item   A pre-compiled binary image of the \emph{kernel}:
                \FileName{uimage};
        \item   An image of the \emph{\BusyBox-based filesystem}:
                \FileName{rootfs.ext2.gz.uboot};
        \item   The \emph{device tree blob}:
                \FileName{mpc8313erdb.dtb}
        \end{itemize}
        These three items are stored by default inside the board flash
        memory (Subsection~\ref{sub:ImagesChecking} explains the technique
        I used to verify this fact).

\end{itemize}

The configurations files they can be found inside the
\FileName{ltib.tar.gz} tarball, in a directory named \FileName{config}.
They can be used in case of partial or total rebuilding of the system.

\Warn{about configurations}{
    If an upgrade to newer versions of the software is required, please
    mind the fact that old configuration files are usually not suitable
    for new versions of the software
}

\subsection{ The cross-compiler }

    The cross-compiler is packaged for a \RedHat\ system, and the
    \StdName{rpm} file uses \FileName{/opt/mtwk} as installation
    directory. On a \Debian\ system you may want to convert it with
    the \TechName{alien} tool into a \StdName{deb} package (it seems also
    safe to install it directly).

    \Warn{Compiler issues?}{
        Before discovering the available \Linux\ kernel image I tried to
        recompile it with the given cross-compiler and kernel
        configuration.  The process gave me some troubles due to a
        ``\emph{internal compiler error}''. This has been verified with
        the 2.6.20 (the version I've found pre-installed on the board).
    }

\subsection{ Other useful tools } \label{sub:OtherTools}

    \begin{itemize}

    \item  The \mkimage\ tool is needed to build binary images:

        \begin{itemize}
        \item   Available for \Debian-like distributions in the
                \PackageName{u-boot-tools} package;
        \item   Available for \RedHat-like distributions in the
                \PackageName{uboot-tools} package (\TechName{EPEL}
                repositories).
        \end{itemize}

    \item   The communication via serial port can be achieved by using
            \Minicom, available for both \Debian-like and \RedHat-like
            distributions as \PackageName{minicom};

    \item   \BootP\ and \TFTP\ daemons are needed, and many solutions are
            available for both of them.
        \begin{itemize}
        \item   I used the server packaged as \PackageName{tftpd}
                under \Debian-like distributions and
                \PackageName{tftp-server} under \RedHat-like
                distributions;
        \item   I used the server packaged as \PackageName{bootp}
                under \Debian-like distributions. I'm not sure about
                the package name under \RedHat.
        \end{itemize}

    \item   To run the script described in Subsection~\ref{sub:Stats}
            you'll need \TechName{Python Matplotlib}, packaged as
            \PackageName{python-matplotlib} for both \Debian-like and
            \RedHat-like distributions.

    \end{itemize}

