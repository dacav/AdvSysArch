The \MyBoard\ is provided with a \uBoot\ software, which by default loads
a \Linux\ kernel image with a \BusyBox\ environment. The command lines of
both systems can be accessed through the serial port. In order to
communicate through the serial port, an \emph{USB to serial} adapter can
be used: on a recent enough \GNULinux\ operating system the procedure
should be totally seamless.

The board comes along with two CD-ROMs providing the
\TechName{CodeWarrior} IDE and some kind of loading system based on the
\LTIB\footnote{ \url{http://www.bitshrine.org/ltib/} } software, however
I'm a suspicious guy, I tend to mistrust anything trying to simplify my
life beyond a certain threshold.  This is true especially when the
software requires to run the \emph{rpm} command with \emph{root}
privileges on my \Debian-like OS!

I ran the program into a \TechName{CentOS} \emph{chroot jail}, but I
still got confused on what the software is supposed to do. Instead of
trying understanding it I preferred to do some retro-engineering of the
system and go for the \emph{bare-metal} approach I'm going to describe in
this document.

The idea is simple: with both \uBoot\ and \Linux\ the Ethernet interfaces
can be enabled, and this can be exploited to conveniently load programs on
the board. The development process can be achieved with the classic
\emph{editor + compiler} combo, which in my opinion is definitely the
simplest way of getting stuff done.

The CD-ROMs however are not totally useless, since one of them contains
some precious files that usually nobody want to produce by themselves.
More on that in the next section.

\subsection{ Naming conventions of this document }

    \begin{itemize}

    \item   The \MyBoard\ will be simply referred as
            ``\emph{board}'';

    \item   The personal computer or laptop on which compilations and
            configurations are achieved is referred as ``\emph{host}'';

    \item   Listings to be executed on the host will have the
            ``\lstinline{host$}'' prompt:
\begin{lstlisting}
    host$ echo Hello
\end{lstlisting}

    \item   Listings to be executed on the board with the
            \BusyBox\ + \Linux\ environment will have the
            ``\lstinline{board$}'' prompt:
\begin{lstlisting}
    board$ echo Hello
\end{lstlisting}

    \item   Listings to be executed on the board with the \uBoot\
            system will have the ``\lstinline{=>}'' prompt:
\begin{lstlisting}
    => echo Hello
\end{lstlisting}

    \end{itemize}

\subsection{ Resources } \label{sub:Resources}

    During my laboratory experience I produced a bunch of files
    (configurations, patches and this document). Everything has been
    stored on my \TechName{Github} repository at
    \url{https://github.com/dacav/AdvSysArch}. You can both download a
    bundle containing everything or install \TechName{git} and run
\begin{lstlisting}
    host$ git clone git://github.com/dacav/AdvSysArch.git
\end{lstlisting}

\subsection{ What are we running? }

    During the process realization I aimed at time efficiency:
    instead of producing from scratch all the tools I preferred to
    hack and modify the existent environment.

    The approach is reasonable, since the system was not that
    outdated as I initially thought, anyways one might want to
    update it. The information contained in this document can be
    very useful also for this kind of operation.

