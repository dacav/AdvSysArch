After a proper testing phase one may want to deploy the application in a
stable environment, possibly with some automatisms.

An Embedded \Linux\ environment needs three main items: \emph{kernel}, a
\emph{filesystem image} and a \emph{device tree}. Those three guys, as
explained in Section~\ref{sec:GetTools}, are provided by
\TechName{Freescale}.

If we want to permanently install programs on the board we basically need
to modify the filesystem, while the other two elements can be used as they
are (as long as the functionality we need aren't placed in kernel space).

\subsection{The filesystem image}

    The filesystem is named \FileName{rootfs.ext2.gz.uboot}, thus what I
    expected is just a \emph{gzipped} \StdName{ext2} filesystem enveloped
    into some \uBoot\ wrapper:

\begin{lstlisting}
    host$ file rootfs.ext2.gz.uboot
    rootfs.ext2.gz.uboot: u-boot legacy uImage, uboot ext2 ramdisk
    rootfs, Linux/PowerPC, RAMDisk Image (gzip), 2831355 bytes,
    Fri Aug 24 17:01:41 2007, Load Address: 0x00000000, Entry Point:
    0x00000000, Header CRC: 0x9B7D6AEB, Data CRC: 0x14B719EB
\end{lstlisting}

    Searching on the web I've found out that this kind of image is
    produced by the \mkimage\ tool, which envelops stuff inside a 64 bytes
    header. Not a big deal for our friend \Command{dd}, right?

    So this is an example shell session you can use to mount the
    filesystem on some directory:
\begin{lstlisting}
    host$ dd if=rootfs.ext2.gz.uboot bs=64 skip=1 of=rootfs.ext2.gz
    44239+1 records in
    44239+1 records out
    2831355 bytes (2.8 MB) copied, 0.0881506 s, 32.1 MB/s
    host$ gunzip rootfs.ext2.gz
    host$ mkdir userland
    host$ su -c 'mount -t ext2 -o rw,loop rootfs.ext2 userland'
    Password:
    host$ ls userland/
    bin/  etc/   lib/      lost+found/  opt/   root/  sys/  usr/
    dev/  home/  linuxrc@  mnt/         proc/  sbin/  tmp/  var/
\end{lstlisting}
    Bingo!

    \Note{Using \Command{su} or \Command{sudo}}{
        Depending on your distribution's policy, the execution of a
        command with \emph{root} privileges could require the \Command{su}
        or the \Command{sudo} command. For instance, if you are on a
        \TechName{Ubuntu} system you are likely to be using \Command{sudo}
        for the \Command{mount} command:
        \begin{quote}\texttt{
            host\$ sudo mount -t ext2 -o rw,loop rootfs.ext2 userland
        }
        \end{quote}
    }

    Now we need to install some program, say the \emph{hello world}
    program we compiled in Section~\ref{sec:Upload}, inside the image. We
    just need to copy it:

\begin{lstlisting}
    host$ cp /tmp/hello userland/bin
\end{lstlisting}

\subsection{The startup script}

    At this point we need a little magic: during bootstrap the \Linux\
    kernel relies on a temporary root filesystem, which usually contains
    the modules to be loaded before starting the actual operating system.
    When the modules are loaded the \Command{pivot\_root} command is used
    to swap the temporary root with the actual root.

    Many different solutions are supported here, but the
    \StdName{initramfs} is definitely the most easy and immediate: it's
    shaped as a \StdName{gzipped cpio} archive. More information on this
    topic can be easily found on the web.

    After the bootstrap, the \Linux\ kernel will try to execute the file
    named \FileName{/init}, so what we want to do is skip the root
    pivoting and remain in the \StdName{initramfs} environment. If we want
    to automatically execute the \emph{hello world} program at bootstrap
    we just need something like this:

\begin{lstlisting}
    host$ cat > userland/init << EOF
    > #!/bin/ash
    >
    > mount -t proc /proc proc
    > mount -t sysfs none /sys
    >
    > hello
    >
    > echo 'Ready to go! Welcome!'
    > /bin/ash --login
    > EOF
\end{lstlisting}

    Now we can build an \StdName{initramfs} image starting from the
    modified filesystem by using the \Command{cpio} command, and
    subsequently envelop it into the wrapper for \uBoot\ by using mkimage

\begin{lstlisting}
    host$ cd userland
    host$ find . | cpio -o -H newc | gzip --best > ../initramfs.img
    host$ cd ..
    host$ mkimage -A powerpc -O Linux -T ramdisk -C gzip -n Sylvie -a 0 -e
          0 -d initramfs.img rootfs.hello.uboot

    Image Name:   Sylvie
    Created:      Wed Jul 27 17:31:47 2011
    Image Type:   PowerPC Linux RAMDisk Image (gzip compressed)
    Data Size:    2800916 Bytes = 2735.27 kB = 2.67 MB
    Load Address: 00000000
    Entry Point:  00000000
\end{lstlisting}

    Our image, the file \FileName{rootfs.hello.uboot} is now ready to be
    loaded on the board. This aspects are covered in
    Section~\ref{sec:UploadImages}.

