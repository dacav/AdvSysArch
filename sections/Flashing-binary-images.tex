The \uBoot\ software supports the network booting through \BootP\ + \TFTP.
This simplifies a lot the transfer of binary images: just setup servers
and ask for data from the board!

\subsection{ Network setup with \uBoot }

    The board is provided with two network interfaces: the
    \TechName{Vitesse VSC7385} and the \TechName{Marvell}. In
    \uBoot\ those are respectively enumerated as \Const{TSEC0} and
    \Const{TSEC1}, and only one of them at a time can be
    activated. This setting is bound by the environment variable
    named \Const{ethact}\footnote{
        Thanks to Lorenzo and Mirko for this information!
    }.

    In order to select the \TechName{Marvell} card the following
    command is needed:
\begin{lstlisting}
    => setenv ethact TSEC1
\end{lstlisting}

    \Note{Variable reset problem}{
        As specified by the \uBoot\ manual, the internal
        \uBoot\ environment can be saved by using the
        \Command{saveenv} command. This works correctly for any
        variable including \Const{ethact}, but for some reason
        the active NIC gets re-assigned when \uBoot\ starts or
        gets reset.

        This is not a problem if you plan to use \BootP\ for
        booting the system.
    }

    During my experiments I noticed that \uBoot\ allows to set an
    IP address for the board, thus I skipped the configuration of
    the \BootP\ daemon and I relied on the default address of the
    NIC.

    The status of the network connection can be verified by using
    the simple version of the \emph{ping} program embedded by
    \uBoot: this works from the board to a host on the network,
    while the vice-versa won't work. By running a \emph{sniffer}
    I discovered that \uBoot\ don't answer to \emph{arp
    requests}, and this must be taken in account in order to
    achieve a network communication: the MAC address of the board
    must be specified on the host running the \TFTP\ server. This
    can be done by using the \Command{arp} program in this way:
\begin{lstlisting}[language=bash]
    arp -s 192.168.1.100 00:12:34:56:78
\end{lstlisting}

    \Note{How to deal with \TechName{network-manager}}{
        The \TechName{network-manager} program, very common on
        recent \GNULinux\ distributions, deactivates Ethernet
        interfaces on which there's no carrier. This can be
        obnoxious, since the \emph{ARP cache} will be cleaned each
        time you reset the board. You may want to temporarily
        deactivate \TechName{network-manager} or manually set the
        \emph{ARP} address when needed.
    }

\subsection{ \TFTP\ repository }

    The \TFTPd\ daemon is packaged for \emph{Debian}-like
    distributions as \PackageName{tftpd} and \emph{Red-Hat}-like
    distributions as \PackageName{tftp-server}. In both cases it
    runs against the \TechName{xinetd} server\footnote{
        \url{http://xinetd.org}
    }, thus the configuration must be entered into the
    \FileName{/etc/xinetd.d} directory. This is the content of my
    \FileName{/etc/xinetd.d/tftp} file:

\begin{lstlisting}
    service tftp
    {
        id = tftp-dgram
        type = UNLISTED
        disable = no
        socket_type = dgram
        protocol = udp
        wait = yes
        server = /usr/sbin/in.tftpd
        server_args = -s /opt/tftpd
        per_source = 1
        user = root
        port = 69
    }
\end{lstlisting}

    where \FileName{/opt/tftpd} is the directory where I stored
    all the binary objects mentioned in
    Section~\ref{sec:GetTools}.

    Once the server has been configured, the \TechName{xinetd}
    daemon must be restarted in order for the modifications to
    take place.

\subsection{ Installing binary objects }


