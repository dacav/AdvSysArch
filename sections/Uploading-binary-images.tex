The \uBoot\ software supports the network booting through \BootP\ + \TFTP.
This functionality is extremely useful, since it allows to transfer a
binary object, like a \emph{filesystem image} or a \emph{kernel}, from a
server to the board.

The default system, installed by \TechName{Freescale}, defines a boot
procedure which loads an embedded operating system from the flash memory.
I modified such setting, and now it behaves as follows:

\begin{enumerate}

\item   After a basic bootstrap the board queries the network with the
        \StdName{dhcp};

\item   The host answers providing:
    \begin{itemize}
    \item   An IP address;
    \item   The name of a \uBoot\ script to be loaded and executed.
    \end{itemize}

\item   Basing on the script two different things can happen:
    \begin{itemize}
    \item   The default system booting procedure is activated
            (\emph{normal booting});
    \item   A new \emph{initramfs} is loaded into memory and copied inside
            the flash memory, replacing the previous image of the
            filesystem (\emph{update booting}).
    \end{itemize}
    In both cases the assigned IP number is propagated also to the \Linux\
    system trought the kernel's command line options.

\end{enumerate}

This procedure particularly well suited for a network environment: by
modifying the settings of the \BootP\ + \TFTP\ server it's possible to
redefine the behavior of the deployed boards in a centralized way.

The next paragraphs will describe in detail how the system works and gets
built.

\subsection{ Network setup }

    The board is provided with two network interfaces: the
    \TechName{Vitesse VSC7385} and the \TechName{Marvell}. In
    \uBoot\ those are respectively enumerated as \Const{TSEC0} and
    \Const{TSEC1}, and only one of them at a time can be
    activated. This setting is bound by the environment variable
    named \Const{ethact}\footnote{
        Thanks to Lorenzo and Mirko for this information!
    }.

    In order to select the \TechName{Marvell} card the following
    command is needed:
\begin{lstlisting}
    => setenv ethact TSEC1
\end{lstlisting}

    \Warn{Variable reset problem}{
        As specified by the \uBoot\ manual, the internal
        \uBoot\ environment can be saved by using the
        \Command{saveenv} command. This works correctly for any
        variable including \Const{ethact}, but for some reason
        the active NIC gets re-assigned when \uBoot\ starts or
        gets reset.
    }

    During my experiments I noticed that \uBoot\ allows to both set
    manually an IP address for the board or use a \BootP\ server in order
    to make the procedure automated. The status of the network connection
    can be verified by using a simple version of the \emph{ping} program
    embedded into \uBoot.

    By running a \emph{sniffer} I discovered that \uBoot\ doesn't answer
    to \emph{arp requests}: the MAC address must be setted manually on the
    host. This problem doesn't show up if a \BootP\ server is used.

    \Note{How to deal with \TechName{network-manager}}{
        The \TechName{network-manager} program, very common on
        recent \GNULinux\ distributions, deactivates Ethernet
        interfaces on which there's no carrier. This can be
        obnoxious, since the \emph{ARP cache} will be cleaned each
        time you reset the board. You may want to temporarily
        deactivate \TechName{network-manager} or manually set the
        \emph{ARP} address when needed.
    }

\subsection{ \BootP\ and \TFTPd } \label{sub:Xinetd}

    Many tutorials on the Internet cover in detail the setup the needed
    for both daemons. Here I'll just show some of my configuration files
    and explain how I'm using them.

    Both the choosen versions of the packages (see
    Subsection~\ref{sub:OtherTools}) rely on \TechName{Xinetd}\footnote{
        \url{http://xinetd.org}
    }, a secure replacement for the \TechName{Inet} daemon.

    \subsubsection{\TFTPd}

\begin{lstlisting}
    host$ more /etc/xinetd.d/tftp 
    service tftp
    {
        id = tftp-dgram
        type = UNLISTED
        disable = no
        socket_type = dgram
        protocol = udp
        wait = yes
        server = /usr/sbin/in.tftpd
        server_args = -s /opt/tftpd
        per_source = 1
        user = root
        port = 69
    }
\end{lstlisting}

        My own setting uses the directory \FileName{/opt/tftpd} has been
        chosen as root for the server. All the binaries mentioned in
        Section~\ref{sec:GetTools} are stored there.

    \subsubsection{\BootP}

\begin{lstlisting}
    host$ more /etc/xinetd.d/bootp 
    service bootps
    {
        id = bootp-dgram
        type = UNLISTED
        disable = no
        socket_type = dgram
        protocol = udp
        wait = yes
        user = root
        server = /usr/sbin/bootpd
        per_source = 1
        port = 67
    }
\end{lstlisting}

        The \BootP\ daemon requires a further configuration effort: the
        \FileName{/etc/bootptab} file must also contain the IP address and
        other booting information. Again, here I'll just show my
        configuration, while further details on how such configuration
        behaves are given later in this document:

\begin{lstlisting}
    host$ more /etc/boottab
    .default:\
        :sm=255.255.255.0:\
        :sa=192.168.1.1:\
        :gw=192.168.1.1

    .noupdate:\
        :tc=.default:\
        :bf=null:\
        :bf=bootseq_run.img

    .update:\
        :tc=.default:\
        :bf=bootseq_run.img

    board:\
        :ht=ether:\
        :ha=00fa10fafa10:\
        :tc=.update:\
        :ip=192.168.1.200
\end{lstlisting}

