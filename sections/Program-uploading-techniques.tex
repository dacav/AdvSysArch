\subsection{ Quick \& Ephemeral }

    When I booted the board with the untampered factory settings I
    noticed that the installed \BusyBox\ + \Linux\ system is
    provided with a \NetCat-like tool\footnote{
        \url{http://nc110.sourceforge.net/}
    }.

    The very first test I ran was trying to execute a simple
    \emph{Hello World} C program on the board as follows:
    \begin{enumerate}
    \item   Cross-compile the program:
\begin{lstlisting}
    host$ cat > hello.c << EOF
    > #include <stdio.h>
    > int main ()
    > {
    >     printf("Hello world\n");
    >     return 0;
    > }
    > EOF
    host$ $ CC=powerpc-linux-gcc make hello
    powerpc-linux-gcc   hello.o   -o hello
\end{lstlisting}
    \item   Feed \NetCat\ with the executable:
\begin{lstlisting}
    host$ nc -l 9000 < hello
\end{lstlisting}
    \item   Retrieve the program from the board and execute it:
\begin{lstlisting}
    board$ nc 192.168.1.1 > hello &
    board$ chmod +x ./hello
    board$ ./hello
    Hello world
\end{lstlisting}
    \end{enumerate}

    This technique is a very simple way to run a program on the
    board, but since the filesystem works in RAM the program will
    be lost on the first reboot of the board.

\subsection{ Slow \& Permanent }

    \subsubsection{ Hacking the filesystem image }

    \subsubsection{ Rebuilding the \InitRamFs }


