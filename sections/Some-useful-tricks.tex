\subsection{ Images checking } \label{sub:ImagesChecking}

    When I first analyzed the original system on the board, I
    wasn't sure about which of the binary images were stored
    inside the flash memory.

    Of course the file names are self-explanatory enough, but the
    CD-ROM provides many different files of the same type. There
    are, for instance, many \emph{device tree blobs}:
    \begin{itemize}
        \item \FileName{mpc8313erdb.dtb}
        \item \FileName{mpc8313erdb\_usbgadget\_external\_phy.dtb}
        \item \FileName{mpc8313erdb\_usbgadget\_internal\_phy.dtb}
        \item \FileName{mpc8313erdb\_usbhost\_external\_phy.dtb}
        \item \FileName{mpc8313erdb\_usbotg\_external\_phy.dtb}
    \end{itemize}
    How to choose the right one?

    The \uBoot\ system provides a command named
    \Command{crc}\footnote{
        \url{http://www.denx.de/wiki/view/DULG/UBootCmdGroupMemory\#Section\_5.9.2.2.}
    }
    which as computes the \StdName{CRC-32 checksum} of a certain
    memory area.

    The size of all \emph{.dtb} files is 12 Kilobytes
    (\Const{0x3000} in hexadecimal) thus, once
    the address of the installed \emph{device tree} is known, its
    \StdName{CRC} can be obtained as follows:
\begin{lstlisting}
    => crc fe300000 0x3000
    CRC32 for fe300000 ... fe302fff ==> cb588b5e
\end{lstlisting}

    Now we can use a checksum program (like \Command{jacksum}) to
    obtain the \StdName{CRC-32} of all images:
\begin{lstlisting}
    host$ for i in *dtb; do jacksum -a crc32 $i; done
    2678061774	12288	mpc8313erdb.dtb
    1614041190	12288	mpc8313erdb_usbgadget_external_phy.dtb
    161761636	12288	mpc8313erdb_usbgadget_internal_phy.dtb
    2371930621	12288	mpc8313erdb_usbhost_external_phy.dtb
    3656584488	12288	mpc8313erdb_usbotg_external_phy.dtb
\end{lstlisting}
    If we convert in decimal the value extracted from \uBoot\ we
    get:
\begin{lstlisting}
    host$ echo 'print 0x9f9fface, "\n"' | perl
    2678061774
\end{lstlisting}
    which corresponds to the \FileName{mpc8313erdb.dtb} file.

    This trick can be used to check consistency of uploaded
    images.

